\documentclass[runningheads]{llncs}

% SETUP
\usepackage{graphicx}
\usepackage{booktabs}
% Gets rid of font shapes not available warnings
\usepackage[utf8]{inputenc}
% intuitive quotation marks
\usepackage{csquotes}
% uses files for each chapter
\usepackage{subfiles}
% If you use the hyperref package, please uncomment the following line
% to display URLs in blue roman font according to Springer's eBook style:
\usepackage{hyperref}
\usepackage{color}
\renewcommand\UrlFont{\color{blue}\rmfamily}


\begin{document}

% TITLE
\title{Corpora Evaluation and System Bias Detection in Multi-document Summarization}
\subtitle{Impact \& Alternatives}
\titlerunning{Corpora Evaluation and System Bias Detection}
\institute{University of Bamberg, Bamberg, Germany\\
    \email{yannick-stephan.lang@stud.uni-bamberg.de}}
\author{Yannick Lang \\
    \small Study Program: B.Sc. Applied Computer Science }
\authorrunning{Y. Lang}

\maketitle              % typeset the header of the contribution


% ABSTRACT
\begin{abstract}
    Large language models are on the rise and as such have an increasing contact surface with more and more people. Partly because they are trained on data produced by people, systems that generate language may reproduce or even insert bias. This paper focuses on metrics for quantifying bias in multi-document summarization corpora and systems, commonly used to compress news articles or similar area where multiple documents deal with a common topic. Different proposals for such metrics will be discussed and their usage assessed. In the end, the \enquote{Wikipedia Current Events Portal} corpus will be benchmarked on selected metrics.
    
    \keywords{Bias \and Summarization \and Corpora \and Language Models}
\end{abstract}

% SECTIONS
\section{Introduction}
\label{sec:introcution}
\subfile{sections/10_introduction}

\section{Original Paper}
\label{sec:original-paper}
\subfile{sections/20_original-paper}

\section{Alternative Criteria \& Status Quo}
\label{sec:alternative-criteria}
\subfile{sections/40_alternative-criteria}

\section{Own Analysis}
\label{sec:analysis}
\subfile{sections/60_analysis}

\section{Summary}
\label{sec:summary}
\subfile{sections/70_summary}


% BIBLIOGRAPHY
\bibliographystyle{splncs04}
\bibliography{references}
\end{document}
