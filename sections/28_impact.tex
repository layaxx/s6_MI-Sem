\documentclass[20_original-paper.tex]{subfiles}
\begin{document}

Now that the benefits and use cases of universally accepted and reported metrics for MDS corpus and system attributes haven been established, it is time to look at the impact of the paper. Did the other scientists build upon these proposed metrics or apply them to their own system or corpus?

To answer this question, forward search was performed via the online research tool SemanticScholar.org.
This reveals publications that reference "Corpora Evaluation and System Bias detection in Multi Document Summarization", which allows an assessment of its impact in the scientific world.

During the almost two years since its publication in October 2020, the paper was referenced 5 times. SemanticScholar.org classifies those citations into three categories: \cite{Cohan2019StructuralSF}

\begin{itemize}
    \item Background - 3 citations
    \item Methods - 1 citation
    \item Results - 1 citation
\end{itemize}

It also identifies one citation, in "ConvoSumm: Conversation Summarization Benchmark and Improved Abstractive Summarization with Argument Mining" (ConvoSumm)\cite{Fabbri2021ConvoSummCS},
as "highly influential"\cite{Valenzuela2015IdentifyingMC}.
ConvoSumm propose a benchmark of summaries and performances for four "widely-used datasets".
The authors use the metrics for Inter-document Similarity, Redundancy and Layout Bias proposed in the original paper to evaluate their data.
They also publish the results and their key finding from applying the metrics, which is a reduced layout bias in their data compared with the mostly news focused datasets that were used in the original paper.
While they do not use every metric, their choice seems reasonable, but deviates from the original papers authors assessment of the metrics importance, as
Dey et al. consider Pyramid Score and Inverse Pyramid score the most important metrics to report.\cite{dey-etal-2020-corpora}

The second citation is in a recently published paper called "Two-phase Multi-document Event Summarization on Core Event Graphs" \cite{Chen2022TwophaseME}. The authors also uses a selection of the proposed metrics to evaluate their datasets, namely abstractness and IDS.

The other three papers reference the original paper in a minor way, usually to back up general statements or disclaimers about MDS corpora potentially containing bias or the possibility of bias propagating to the generated content.\cite{Moro2022SemanticSF,Adams2021WhatsIA,Counseling}

An overview over the types of citations is given in \ref{tab:impact}.

In summary, it can be stated that the original paper has not yet had a lot of impact, which is not all that surprising given its somewhat recent publishing at the end of 2020.
Two of the five studies that reference this paper actually use the proposed metrics, however it stands out that neither of those use the two metrics the original authors deemed most important.

\begin{table}[]
    \centering
    \begin{tabular}{lllll}
        \multicolumn{5}{c}{uses metrics}                                                                                                                                            \\ \hline
        \multicolumn{1}{l|}{}                                       & \rotatebox{90}{IDS} & \rotatebox{90}{Abstractness} & \rotatebox{90}{Layout Bias} & \rotatebox{90}{Redundancy} \\ \hline
        \multicolumn{1}{l|}{ConvoSumm \cite{Fabbri2021ConvoSummCS}} & x                   &                              & x                           & x                          \\
        \multicolumn{1}{l|}{2-Phase \cite{Chen2022TwophaseME}}      & x                   & x                            &                             &                            \\ \hline
    \end{tabular}
    \quad
    \begin{tabular}{ll}
        \multicolumn{2}{c}{generic citation}                                                \\ \hline
        \multicolumn{1}{l|}{}                                           & citation for:     \\ \hline
        \multicolumn{1}{l|}{Segmentation \cite{Cohan2019StructuralSF}}  & existence of bias \\
        \multicolumn{1}{l|}{What is a Summary? \cite{Adams2021WhatsIA}} & existence of bias \\
        \multicolumn{1}{l|}{Counseling \cite{Counseling}}               & past studies      \\ \hline
    \end{tabular}
    \caption{Breakdown of the context in which the five papers reference the original paper. In the left table, only metrics that were used are displayed, the x character indicates that this paper used this metric.}
    \label{tab:impact}
\end{table}

\end{document}