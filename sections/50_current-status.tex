\documentclass[../main.tex]{subfiles}
\begin{document}

In this section, the status quo shall be outlined, i.e. how researchers, that propose MDS corpora handle bias and the overall quality of their dataset.

As previously stated, there is no widely accepted framework for this, leading to different approaches by different authors.
I have selected recent (published in 2020 or later) papers introducing MDS corpora, and will now show their approach to bias and comparability.

\begin{table}
    \centering
    \begin{tabular}{l|lll}
                                                                                 & mention of bias               & metrics reported                   & \\
        \cline{1-3}
        \textbf{MS2}\cite{MS2-https://doi.org/10.48550/arxiv.2104.06486}         & no                            & basic statistics                   & \\
        \textbf{HowSumm}\cite{wikihow-https://doi.org/10.48550/arxiv.2110.03179} & no                            & basic statisics, coverage  density & \\
        \textbf{QMSumm}\cite{qmsum-https://doi.org/10.48550/arxiv.2104.05938}    & mentioned, but not quantified & basic statistics                   & \\
        \multicolumn{1}{l}{}                                                     &                               &                                    &
    \end{tabular}

    \caption{Recent papers introducing MDS corpora and their handling of bias/comparability. Basic statistics refers to metrics such as count and average length}
    \label{tab:status-quo}
\end{table}



% https://arxiv.org/pdf/2106.00829.pdf -> ConvoSumm
% TODO: look at this: Bias may be propagated

% https://paperswithcode.com/paper/a-large-scale-multi-document-summarization A Large-Scale Multi-Document Summarization Dataset from the Wikipedia Current Events Portal -> not mention of bias or metrics, available for download
% TODO: this could be a good transition to next section


\end{document}