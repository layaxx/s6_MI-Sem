\documentclass[20_original-paper.tex]{subfiles}
\begin{document}

The paper proposes a list of metrics which can be used to gauge certain attributes of a corpus. These metrics are:

\begin{itemize}
    \item \textbf{Inter Document Similarity (IDS)}, which shows the similarity between each documents
    \item \textbf{Pyramid Score}, \enquote{defined as the ratio of a reference summary score and an optimal summary score}\cite{dey-etal-2020-corpora},
          i.e. how good the reference summaries are
    \item \textbf{Inverse Pyramid Score}, which measures the influence of documents on a the reference summaries
    \item \textbf{Redundancy} describes the density of information in the documents
\end{itemize}

For MDS systems, they also present a list of metrics aimed at evaluating the performance and properties of a given MDS systems:

\begin{itemize}
    \item \textbf{ROUGE} one of the most basic metrics in text summarization,
          which has been used to measure similarity between generated summaries and the references included in the dataset via recall since 2004\cite{lin-2004-rouge}
    \item \textbf{F1 Score} - similar to ROUGE but considers both recall and precision
    \item \textbf{Inter Document Distribution} - similarly to the Inverse Pyramid Score metric for corpora, this measures the influence of each document on the generated summary
    \item \textbf{Redundancy} describes the summaries coverage of information from the documents

\end{itemize}

Additionally, some metrics are suitable for both both corpus and system, where the reference summary is used for corpus evaluation and the generated summary for system evaluation.

\begin{itemize}
    \item \textbf{Abstractness} quantifies the similarity between the generated or reference summary and the associated documents, where less similarity means higher abstractness
    \item \textbf{Layout Bias} measures the distribution of information within a document for corpora and the distribution of sections in the documents that provide the information in the generated summary for systems
\end{itemize}

The authors categorize those metrics into subjective and objective.
They assign the highest importance to the objective metrics of Pyramid Score and Inverse Pyramid Score,
and propose that scientists introducing new MDS corpora shall at least report values for those, although ideally all metrics should be considered.

\end{document}